% Template:     Template Auxiliar LaTeX
% Documento:    Archivo principal
% Versión:      4.4.3 (05/02/2018)
% Codificación: UTF-8
%
% Autor: Pablo Pizarro R.
%        Facultad de Ciencias Físicas y Matemáticas
%        Universidad de Chile
%        pablo@ppizarror.com
%
% Sitio web:    [http://latex.ppizarror.com/Template-Auxiliares/]
% Licencia MIT: [https://opensource.org/licenses/MIT]

% CREACIÓN DEL DOCUMENTO
\documentclass[letterpaper,11pt]{article} % Articulo tamaño carta, 11pt
\usepackage[utf8]{inputenc} % Codificación UTF-8

% INFORMACIÓN DEL DOCUMENTO
\def\tituloauxiliar {Cheat Sheet Numpy/Matplotlib Matlab}
\def\temaatratar {Guía apoyo}

\def\autordeldocumento {Pablo Pizarro R.}
\def\nombredelcurso {Modelación y Computación Gráfica para Ingenieros}
\def\codigodelcurso {CC3501-1}

\def\nombreuniversidad {Universidad de Chile}
\def\nombrefacultad {Facultad de Ciencias Físicas y Matemáticas}
\def\departamentouniversidad {Departamento de Ciencias de la Computación}
\def\imagendepartamento {departamentos/dcc}
\def\imagendepartamentoescala {0.2}
\def\localizacionuniversidad {Santiago, Chile}

% EQUIPO DOCENTE
\def\equipodocente {
\begin{centering}\begin{tabular}{lll}
	\begin{tabular}[t]{@{}c@{}}
		Pablo Pizarro R. \\
		\insertemail{pablo@ppizarror.com}
	\end{tabular}
\end{tabular}\end{centering}
}

% CONFIGURACIONES
\input{lib/config}

% IMPORTACIÓN DE LIBRERÍAS
\input{lib/imports}

% IMPORTACIÓN DE FUNCIONES
\input{lib/function/auxiliar}
\input{lib/function/core}
\input{lib/function/elements}
\input{lib/function/equation}
\input{lib/function/image}
\input{lib/function/title}

% IMPORTACIÓN DE ESTILOS
\input{lib/styles}

% CONFIGURACIÓN INICIAL DEL DOCUMENTO
\input{lib/initconf}

% INICIO DE LAS PÁGINAS
\begin{document}

% CONFIGURACIÓN DE PÁGINA Y ENCABEZADOS
\input{lib/pageconf}

% ======================= INICIO DEL DOCUMENTO =======================

Numpy y Matplotlib son librerías en Python que sirven para realizar operaciones matemáticas tanto de manejo de matrices y vectores como de representación de gráficos, respectivamente.

\newp Para instalar dichas librerías en Linux sólo basta ingresar en el terminal los siguientes comandos:

\lstset{style=Python}
\begin{lstlisting}[language=Python]
sudo pip install numpy
sudo pip install matplotlib
\end{lstlisting}

\newp En caso de estar utilizando Windows, a partir de las versiones \texttt{2.7.9} para python 2 y \texttt{3.4} para python 3, pip viene incluido y se ejecuta con las instrucciones (estando python en el respectivo \texttt{PATH}):

\begin{lstlisting}[language=Python]
python -m pip install numpy
python -m pip install matplotlib
\end{lstlisting}

\newp Finalmente para acceder a las librerías dentro de Python se deben importar haciendo uso de las siguientes instrucciones:

\begin{lstlisting}[language=Python]
import numpy as np
import matplotlib.pylab as pl
\end{lstlisting}

\newp A continuación, comandos básicos de numpy y matlab:

\begin{itemize}
	
\item{
Nan e Infinito:
\begin{lstlisting}[language=Python]
Python: Nan = np.nan; Inf = np.inf
Matlab: _nan = NaN; _inf = Inf
\end{lstlisting}
}
	
\item{
Crear matriz de unos.
\begin{lstlisting}[language=Python]
Python: np.ones((2,3))
	>>	[[1, 1, 1],
		[1, 1, 1]]
Matlab: ones(2,3)
	>>	1 1 1
		1 1 1
\end{lstlisting}
}
	
\newpage
\item{
Crear matriz de ceros.
\begin{lstlisting}[language=Python]
Python: np.zeros((2,3))
	>>	[[0, 0, 0],
		[0, 0, 0]]
Matlab: zeros(2,3)
	>>	0 0 0
		0 0 0
\end{lstlisting}
}
	
\item{
Ponderar matriz.
\begin{lstlisting}[language=Python]
Python: np.ones((2,3)) * 5
	>>	[[5, 5, 5],
		[5, 5, 5]]
Matlab: ones(2,3) .* 5
	>>	5 5 5
		5 5 5
\end{lstlisting}
}
	
\item{
Tamaño matriz.
\begin{lstlisting}[language=Python]
Python: np.zeros((2,3)).shape
	>>	(2, 3)
Matlab: size(zeros(2,3))
	>>	2 3
\end{lstlisting}
}

\item{
Accesores.
\begin{lstlisting}[language=Python]
Python: matriz[a,b] o bien matriz[:,b]
Matlab: matriz(a,b) o bien matriz(:,b)
\end{lstlisting}
}

\item{
Chequeo contra \texttt{NaN}.
\begin{lstlisting}[language=Python]
Python: np.isnan(np.ones((2,3)) * np.nan)
	>>	[[True, True, True],
		[True, True, True]]
Matlab: isnan([1 1 NaN 1 1])
	>>	0 0 1 0 0
\end{lstlisting}
}

\item{
Chequeo contra \texttt{Inf}.
\begin{lstlisting}[language=Python]
Python: np.isinf(np.ones((2,3)) * np.inf)
	>>	[[True, True, True],
		[True, True, True]]
Matlab: isinf([1 1 Inf 1 1])
	>>	0 0 1 0 0
\end{lstlisting}
}

\item{
Elementos distintos de cero.
\begin{lstlisting}[language=Python]
Python: np.nonzero(np.ones((2,3)))
	>>	(array([0, 0, 0, 1, 1, 1]), array([0, 1, 2, 0, 1, 2])) ...
Matlab: [Inf Inf 0; 0 Inf Inf]
	>>	1 1 0
		0 1 1
\end{lstlisting}
}

\newpage
\item{
Flip up-down y left-right.
\begin{lstlisting}[language=Python]
Python: A = np.ones((2,2)) ; A[0,1] = 2; A[1,0] = 3; A[1,1] = 4;
		np.flipud(A)
	>>	[[3, 4],
		[1, 2]]
		np.fliplr(A)
	>>	[[2, 1],
		[4, 3]]
Matlab: flipud([1 2; 3 4])
	>>	3 4
		1 2
		fliplr([1 2; 3 4])
	>>	2 1
		4 3
\end{lstlisting}
}

\item{
Promedio matriz.
\begin{lstlisting}[language=Python]
Python: A = np.ones((2,2)) ; A[0,1] = 2; A[1,0] = 3; A[1,1] = 4;
		np.mean(A) = 2.5
Matlab: mean2([1 2; 3 4]) = 2.5
\end{lstlisting}
}

\item{
Gradiente matriz.
\begin{lstlisting}[language=Python]
Python: A = np.matrix([[5, 2, 9],[8, 5, 1],[2, 1, 8]])
[U,V] = np.gradient(A)
	>>	U = [
			[ 3. 3. 8. ]
			[1.5 0.5 0.5]
			[6. 4. 7.]
		]
	>>	V = [
			[3. 2. 7. ]
			[3. 3.5 4. ]
			[1. 3. 7. ]
		]
Matlab: [U, V] = gradient([5 2 9 ; 8 5 1; 2 1 8])
		U =
			3.0000 2.0000 7.0000
			3.0000 3.5000 4.0000
			1.0000 3.0000 7.0000
		V =
			3.0000 3.0000 8.0000
			1.5000 0.5000 0.5000
			6.0000 4.0000 7.0000
\end{lstlisting}
}

\item{
Mostrar matriz.
\begin{lstlisting}[language=Python]
Python: A = np.matrix([[1, 2, 3],[4, 5, 6],[7, 8, 9]])
		pl.matshow(A)
		pl.show()
Matlab: A = [1 2 3; 4 5 6; 7 8 9]
		surf(A) # o bien mesh(A)
\end{lstlisting}
}

\item{
Trasponer matriz.
\begin{lstlisting}[language=Python]
Python: (a,b) = ([0, 0, 1, 1, 2, 2],[1, 2, 1, 2, 1, 2]) ; ...
		pl.transpose((a,b))
	>>	[
			[0 1]
			[0 2]
			[1 1]
			[1 2]
			[2 1]
			[2 2]
		]
Matlab: transpose([0 0 1 1 2 2; 1 2 1 2 1 2])
	>>	0 1
		0 2
		1 1
		1 2
		2 1
		2 2
\end{lstlisting}
}

\item{
Quiver.
\begin{lstlisting}[language=Python]
# Asumiendo vectores U y V del gradiente.
Python: pl.figure() ; pl.quiver(U,V) ; pl.show()
Matlab: quiver(U,V)
\end{lstlisting}
}
	
\end{itemize}

% FIN DEL DOCUMENTO
\end{document}