% Template:     Template Auxiliar LaTeX
% Documento:    Archivo principal
% Versión:      6.0.9 (16/01/2019)
% Codificación: UTF-8
%
% Autor: Pablo Pizarro R. @ppizarror
%        Facultad de Ciencias Físicas y Matemáticas
%        Universidad de Chile
%        pablo@ppizarror.com
%
% Sitio web:    [https://latex.ppizarror.com/Template-Auxiliares/]
% Licencia MIT: [https://opensource.org/licenses/MIT]

% CREACIÓN DEL DOCUMENTO
\documentclass[letterpaper,11pt]{article} % Articulo tamaño carta, 11pt
\usepackage[utf8]{inputenc} % Codificación UTF-8

% INFORMACIÓN DEL DOCUMENTO
\def\tituloauxiliar {Cheat Sheet Numpy/Matplotlib Matlab}
\def\temaatratar {Guía apoyo}

\def\autordeldocumento {Pablo Pizarro R.}
\def\nombredelcurso {Modelación y Computación Gráfica para Ingenieros}
\def\codigodelcurso {CC3501-1}

\def\nombreuniversidad {Universidad de Chile}
\def\nombrefacultad {Facultad de Ciencias Físicas y Matemáticas}
\def\departamentouniversidad {Departamento de Ciencias de la Computación}
\def\imagendepartamento {dcc}
\def\imagendepartamentoescala {0.2}
\def\localizacionuniversidad {Santiago, Chile}

% EQUIPO DOCENTE
\def\equipodocente {
Pablo Pizarro R. \\
\insertemail{pablo@ppizarror.com}
}

% CONFIGURACIONES
\input{lib/config}

% IMPORTACIÓN DE LIBRERÍAS
\input{lib/env/imports}

% IMPORTACIÓN DE FUNCIONES
\input{lib/cmd/all}

% IMPORTACIÓN DE ESTILOS
\input{lib/style/all}

% CONFIGURACIÓN INICIAL DEL DOCUMENTO
\input{lib/cfg/init}

% INICIO DE LAS PÁGINAS
\begin{document}

% CONFIGURACIÓN DE PÁGINA Y ENCABEZADOS
% Template:     Informe/Reporte LaTeX
% Documento:    Configuración de página
% Versión:      6.2.6 (13/05/2019)
% Codificación: UTF-8
%
% Autor: Pablo Pizarro R. @ppizarror
%        Facultad de Ciencias Físicas y Matemáticas
%        Universidad de Chile
%        pablo.pizarro@ing.uchile.cl, ppizarror.com
%
% Manual template: [https://latex.ppizarror.com/Template-Informe/]
% Licencia MIT:    [https://opensource.org/licenses/MIT/]

\newpage
\ifthenelse{\equal{\predocuseromannumber}{true}}{
	\ifthenelse{\equal{\romanpageuppercase}{true}}{
		\pagenumbering{Roman}
	}{
		\pagenumbering{roman}
	}}{
	\pagenumbering{arabic}
}
\setcounter{page}{1}
\setcounter{footnote}{1}
\setpagemargincm{\pagemarginleft}{\pagemargintop}{\pagemarginright}{\pagemarginbottom}
\def\arraystretch {\tablepaddingv}
\setlength{\tabcolsep}{\tablepaddingh em}
\ifthenelse{\equal{\pointdecimal}{true}}{
	\decimalpoint}{
}
\renewcommand{\appendixname}{\nomltappendixsection}
\renewcommand{\appendixpagename}{\nameappendixsection}
\renewcommand{\appendixtocname}{\nameappendixsection}
\renewcommand{\contentsname}{\nomltcont}
\renewcommand{\figurename}{\nomltwfigure}
\renewcommand{\listfigurename}{\nomltfigure}
\renewcommand{\listtablename}{\nomlttable}
\renewcommand{\lstlistingname}{\nomltwsrc}
\renewcommand{\lstlistlistingname}{\nomltsrc}
\renewcommand{\refname}{\namereferences}
\renewcommand{\tablename}{\nomltwtable}
\sectionfont{\color{\titlecolor} \fontsizetitle \styletitle \selectfont}
\subsectionfont{\color{\subtitlecolor} \fontsizesubtitle \stylesubtitle \selectfont}
\subsubsectionfont{\color{\subsubtitlecolor} \fontsizesubsubtitle \stylesubsubtitle \selectfont}
\titleformat{\subsubsubsection}{\color{\ssstitlecolor} \normalfont \fontsizessstitle \stylessstitle}{\thesubsubsubsection}{1em}{}
\titlespacing*{\subsubsubsection}{0pt}{3.25ex plus 1ex minus .2ex}{1.5ex plus .2ex}
\ifthenelse{\equal{\hfstyle}{style1}}{
	\pagestyle{fancy} \fancyhf{}
	\ifthenelse{\equal{\disablehfrightmark}{false}}{
		\fancyhead[L]{\nouppercase{\rightmark}}
	}{}
	\fancyhead[R]{\small \rm \thepage}
	\fancyfoot[L]{\small \rm \textit{\titulodelinforme}}
	\fancyfoot[R]{\small \rm \textit{\codigodelcurso \nombredelcurso}}
	\renewcommand{\headrulewidth}{0.5pt}
	\renewcommand{\footrulewidth}{0.5pt}
	\renewcommand{\sectionmark}[1]{\markboth{#1}{}}
}{
\ifthenelse{\equal{\hfstyle}{style2}}{
	\pagestyle{fancy} \fancyhf{}
	\ifthenelse{\equal{\disablehfrightmark}{false}}{
		\fancyhead[L]{\nouppercase{\rightmark}}
	}{}
	\fancyhead[R]{\small \rm \thepage}
	\fancyfoot[L]{\small \rm \textit{\titulodelinforme}}
	\fancyfoot[R]{\small \rm \textit{\codigodelcurso \nombredelcurso}}
	\renewcommand{\headrulewidth}{0.5pt}
	\renewcommand{\footrulewidth}{0pt}
	\renewcommand{\sectionmark}[1]{\markboth{#1}{}}
}{
\ifthenelse{\equal{\hfstyle}{style3}}{
	\pagestyle{fancy} \fancyhf{}
	\fancyhead[L]{
		\small \rm \textit{\codigodelcurso \nombredelcurso}
		\vspace{0.04cm}
	}
	\fancyhead[R]{
		\includegraphics[width=1.2cm]{\imagendepartamento}
		\vspace{-0.10cm}
	}
	\fancyfoot[C]{\thepage}
	\renewcommand{\headrulewidth}{0.5pt}
	\renewcommand{\footrulewidth}{0pt}
}{
\ifthenelse{\equal{\hfstyle}{style4}}{
	\pagestyle{fancy} \fancyhf{}
	\ifthenelse{\equal{\disablehfrightmark}{false}}{
		\fancyhead[L]{\nouppercase{\rightmark}}
	}{}
	\fancyhead[R]{}
	\fancyfoot[C]{\small \rm \thepage}
	\renewcommand{\headrulewidth}{0.5pt}
	\renewcommand{\footrulewidth}{0pt}
	\renewcommand{\sectionmark}[1]{\markboth{#1}{}}
}{
\ifthenelse{\equal{\hfstyle}{style5}}{
	\pagestyle{fancy} \fancyhf{}
	\fancyhead[L]{\codigodelcurso \nombredelcurso}
	\ifthenelse{\equal{\disablehfrightmark}{false}}{
		\fancyhead[R]{\nouppercase{\rightmark}}
	}{}
	\fancyfoot[L]{\departamentouniversidad, \nombreuniversidad}
	\fancyfoot[R]{\small \rm \thepage}
	\renewcommand{\headrulewidth}{0pt}
	\renewcommand{\footrulewidth}{0pt}
	\renewcommand{\sectionmark}[1]{\markboth{#1}{}}
}{
\ifthenelse{\equal{\hfstyle}{style6}}{
	\pagestyle{fancy} \fancyhf{}
	\fancyfoot[L]{\departamentouniversidad}
	\fancyfoot[C]{\thepage}
	\fancyfoot[R]{\nombreuniversidad}
	\renewcommand{\headrulewidth}{0pt}
	\renewcommand{\footrulewidth}{0pt}
	\setlength{\headheight}{49pt}
}{
\ifthenelse{\equal{\hfstyle}{style7}}{
	\pagestyle{fancy} \fancyhf{}
	\fancyfoot[C]{\thepage}
	\renewcommand{\headrulewidth}{0pt}
	\renewcommand{\footrulewidth}{0pt}
	\setlength{\headheight}{49pt}
}{
\ifthenelse{\equal{\hfstyle}{style8}}{
	\pagestyle{fancy} \fancyhf{}
	\fancyfoot[R]{\thepage}
	\renewcommand{\headrulewidth}{0pt}
	\renewcommand{\footrulewidth}{0pt}
	\setlength{\headheight}{49pt}
}{
\ifthenelse{\equal{\hfstyle}{style9}}{
	\pagestyle{fancy} \fancyhf{}
	\ifthenelse{\equal{\disablehfrightmark}{false}}{
		\fancyhead[L]{\nouppercase{\rightmark}}
	}{}
	\fancyhead[R]{}
	\fancyfoot[L]{\small \rm \textit{\titulodelinforme}}
	\fancyfoot[R]{\small \rm \thepage}
	\renewcommand{\headrulewidth}{0.5pt}
	\renewcommand{\footrulewidth}{0.5pt}
	\renewcommand{\sectionmark}[1]{\markboth{#1}{}}
}{
\ifthenelse{\equal{\hfstyle}{style10}}{
	\pagestyle{fancy} \fancyhf{}
	\ifthenelse{\equal{\disablehfrightmark}{false}}{
		\fancyhead[L]{\nouppercase{\rightmark}}
	}{}
	\fancyhead[R]{\small \rm \textit{\titulodelinforme}}
	\fancyfoot[L]{}
	\fancyfoot[R]{\small \rm \thepage}
	\renewcommand{\headrulewidth}{0.5pt}
	\renewcommand{\footrulewidth}{0.5pt}
	\renewcommand{\sectionmark}[1]{\markboth{#1}{}}
}{
\ifthenelse{\equal{\hfstyle}{style11}}{
	\pagestyle{fancy} \fancyhf{}
	\ifthenelse{\equal{\disablehfrightmark}{false}}{
		\fancyhead[L]{\nouppercase{\rightmark}}
	}{}
	\fancyhead[R]{\small \rm \thepage \nomnpageof \pageref{LastPage}}
	\fancyfoot[L]{\small \rm \textit{\titulodelinforme}}
	\fancyfoot[R]{\small \rm \textit{\codigodelcurso \nombredelcurso}}
	\renewcommand{\headrulewidth}{0.5pt}
	\renewcommand{\footrulewidth}{0.5pt}
	\renewcommand{\sectionmark}[1]{\markboth{#1}{}}
}{
\ifthenelse{\equal{\hfstyle}{style12}}{
	\pagestyle{fancy} \fancyhf{}
	\fancyfoot[L]{\departamentouniversidad}
	\fancyfoot[C]{\thepage \nomnpageof \pageref{LastPage}}
	\fancyfoot[R]{\nombreuniversidad}
	\renewcommand{\headrulewidth}{0pt}
	\renewcommand{\footrulewidth}{0pt}
	\setlength{\headheight}{49pt}
}{
\ifthenelse{\equal{\hfstyle}{style13}}{
	\pagestyle{fancy} \fancyhf{}
	\fancyhead[L]{
		\small \rm \textit{\codigodelcurso \nombredelcurso}
		\vspace{0.04cm}
	}
	\fancyhead[R]{
		\includegraphics[width=1.2cm]{\imagendepartamento}
		\vspace{-0.10cm}
	}
	\fancyfoot[C]{\thepage \nomnpageof \pageref{LastPage}}
	\renewcommand{\headrulewidth}{0.5pt}
	\renewcommand{\footrulewidth}{0pt}
}{
\ifthenelse{\equal{\hfstyle}{style14}}{
	\pagestyle{fancy} \fancyhf{}
	\ifthenelse{\equal{\disablehfrightmark}{false}}{
		\fancyhead[L]{\nouppercase{\rightmark}}
	}{}
	\fancyhead[R]{}
	\fancyfoot[C]{\small \rm \thepage \nomnpageof \pageref{LastPage}}
	\renewcommand{\headrulewidth}{0.5pt}
	\renewcommand{\footrulewidth}{0pt}
	\renewcommand{\sectionmark}[1]{\markboth{#1}{}}
}{
	\throwbadconfigondoc{Estilo de header-footer incorrecto}{\hfstyle}{style1 .. style14}}}}}}}}}}}}}}
}
\ifthenelse{\equal{\showlinenumbers}{true}}{
	\linenumbers}{
}


% ======================= INICIO DEL DOCUMENTO =======================

Numpy y Matplotlib son librerías en Python que sirven para realizar operaciones matemáticas tanto de manejo de matrices y vectores como de representación de gráficos, respectivamente.

\newp Para instalar dichas librerías en Linux sólo basta ingresar en el terminal los siguientes comandos:

\begin{sourcecode}{bash}{}
sudo pip install numpy
sudo pip install matplotlib
\end{sourcecode}

\newp En caso de estar utilizando Windows, a partir de las versiones \texttt{2.7.9} para python 2 y \texttt{3.4} para python 3, pip viene incluido y se ejecuta con las instrucciones (estando python en el respectivo \texttt{PATH}):

\begin{sourcecode}{bash}{}
python -m pip install numpy
python -m pip install matplotlib
\end{sourcecode}

\newp Finalmente para acceder a las librerías dentro de Python se deben importar haciendo uso de las siguientes instrucciones:

\begin{sourcecode}{python}{}
import numpy as np
import matplotlib.pylab as pl
\end{sourcecode}

\newp A continuación, comandos básicos de numpy y matlab:

\begin{itemize}
	
	\item{
		Nan e Infinito:
		\begin{sourcecode}{python}{}
Python: Nan = np.nan; Inf = np.inf
Matlab: _nan = NaN; _inf = Inf
\end{sourcecode}
	}
	
	\item{
		Crear matriz de unos.
		\begin{sourcecode}{python}{}
Python: np.ones((2,3))
>>	[[1, 1, 1],
[1, 1, 1]]
Matlab: ones(2,3)
>>	1 1 1
1 1 1
\end{sourcecode}
	}
	
	\newpage
	\item{
		Crear matriz de ceros.
		\begin{sourcecode}{python}{}
Python: np.zeros((2,3))
>>	[[0, 0, 0],
[0, 0, 0]]
Matlab: zeros(2,3)
>>	0 0 0
0 0 0
\end{sourcecode}
	}
	
	\item{
		Ponderar matriz.
		\begin{sourcecode}{python}{}
Python: np.ones((2,3)) * 5
>>	[[5, 5, 5],
[5, 5, 5]]
Matlab: ones(2,3) .* 5
>>	5 5 5
5 5 5
\end{sourcecode}
	}
	
	\item{
		Tamaño matriz.
		\begin{sourcecode}{python}{}
Python: np.zeros((2,3)).shape
>>	(2, 3)
Matlab: size(zeros(2,3))
>>	2 3
\end{sourcecode}
	}
	
	\item{
		Accesores.
		\begin{sourcecode}{python}{}
Python: matriz[a,b] o bien matriz[:,b]
Matlab: matriz(a,b) o bien matriz(:,b)
\end{sourcecode}
	}
	
	\item{
		Chequeo contra \texttt{NaN}.
		\begin{sourcecode}{python}{}
Python: np.isnan(np.ones((2,3)) * np.nan)
>>	[[True, True, True],
[True, True, True]]
Matlab: isnan([1 1 NaN 1 1])
>>	0 0 1 0 0
\end{sourcecode}
	}
	
	\item{
		Chequeo contra \texttt{Inf}.
		\begin{sourcecode}{python}{}
Python: np.isinf(np.ones((2,3)) * np.inf)
>>	[[True, True, True],
[True, True, True]]
Matlab: isinf([1 1 Inf 1 1])
>>	0 0 1 0 0
\end{sourcecode}
	}
	
	\item{
		Elementos distintos de cero.
		\begin{sourcecode}{python}{}
Python: np.nonzero(np.ones((2,3)))
>>	(array([0, 0, 0, 1, 1, 1]), array([0, 1, 2, 0, 1, 2])) ...
Matlab: [Inf Inf 0; 0 Inf Inf]
>>	1 1 0
0 1 1
\end{sourcecode}
	}
	
	\newpage
	\item{
		Flip up-down y left-right.
		\begin{sourcecode}{python}{}
Python: A = np.ones((2,2)) ; A[0,1] = 2; A[1,0] = 3; A[1,1] = 4;
np.flipud(A)
>>	[[3, 4],
[1, 2]]
np.fliplr(A)
>>	[[2, 1],
[4, 3]]
Matlab: flipud([1 2; 3 4])
>>	3 4
1 2
fliplr([1 2; 3 4])
>>	2 1
4 3
\end{sourcecode}
	}
	
	\item{
		Promedio matriz.
		\begin{sourcecode}{python}{}
Python: A = np.ones((2,2)) ; A[0,1] = 2; A[1,0] = 3; A[1,1] = 4;
np.mean(A) = 2.5
Matlab: mean2([1 2; 3 4]) = 2.5
\end{sourcecode}
	}
	
	\item{
		Gradiente matriz.
		\begin{sourcecode}{python}{}
Python: A = np.matrix([[5, 2, 9],[8, 5, 1],[2, 1, 8]])
[U,V] = np.gradient(A)
>>	U = [
[ 3. 3. 8. ]
[1.5 0.5 0.5]
[6. 4. 7.]
]
>>	V = [
[3. 2. 7. ]
[3. 3.5 4. ]
[1. 3. 7. ]
]
Matlab: [U, V] = gradient([5 2 9 ; 8 5 1; 2 1 8])
U =
3.0000 2.0000 7.0000
3.0000 3.5000 4.0000
1.0000 3.0000 7.0000
V =
3.0000 3.0000 8.0000
1.5000 0.5000 0.5000
6.0000 4.0000 7.0000
\end{sourcecode}
	}
	
	\item{
		Mostrar matriz.
		\begin{sourcecode}{python}{}
Python: A = np.matrix([[1, 2, 3],[4, 5, 6],[7, 8, 9]])
pl.matshow(A)
pl.show()
Matlab: A = [1 2 3; 4 5 6; 7 8 9]
surf(A) # o bien mesh(A)
\end{sourcecode}
	}
	
	\item{
		Trasponer matriz.
		\begin{sourcecode}{python}{}
Python: (a,b) = ([0, 0, 1, 1, 2, 2],[1, 2, 1, 2, 1, 2]) ; ...
pl.transpose((a,b))
>>	[
[0 1]
[0 2]
[1 1]
[1 2]
[2 1]
[2 2]
]
Matlab: transpose([0 0 1 1 2 2; 1 2 1 2 1 2])
>>	0 1
0 2
1 1
1 2
2 1
2 2
\end{sourcecode}
	}
	
	\item{
		Quiver.
		\begin{sourcecode}{python}{}
# Asumiendo vectores U y V del gradiente.
Python: pl.figure() ; pl.quiver(U,V) ; pl.show()
Matlab: quiver(U,V)
\end{sourcecode}
	}
	
\end{itemize}

% FIN DEL DOCUMENTO
\end{document}