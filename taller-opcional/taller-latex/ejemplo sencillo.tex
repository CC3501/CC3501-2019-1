\documentclass[]{article}

%opening
\title{Taller}
\author{Pablo}

\usepackage{graphicx}
\usepackage{float}
\usepackage{bigstrut}
\usepackage{calculus}
\usepackage{calculator}

\newcommand{\insertadoge}[2]{
	% #1: Porte de la figura
	% #2: Numero del dogecito
	\begin{figure}[H]
		\centering
		\includegraphics[width=#1]{doge.jpg}
		\caption{Doge hermoso variable #2}
	\end{figure}
}

\usepackage{amsmath}
\newcommand{\insertaDogeEnCaja}[2]{
	\begin{figure}[H]
		\centering
		\boxed{\includegraphics[width=#1]{doge.jpg}}
		\caption{Doge hermoso variable #2}
	\end{figure}
}

\newcommand{\insertaLegionDeDoges}[1]{
	\insertaDogeEnCaja{#1}{1}
	\insertaDogeEnCaja{#1}{2}
	\insertaDogeEnCaja{#1}{3}
	\insertaDogeEnCaja{#1}{4}
	\insertaDogeEnCaja{#1}{5}
}

\begin{document}

\maketitle

\begin{abstract}
	
	Texto en el abstract

\end{abstract}

\section{Hice esta ppt a las 4 de la madrugada}

Que facil es latex

\subsection{Hola mundo}

hola latex

\subsection{agrega imagen}

\begin{figure}[H]
	\centering
	\includegraphics[width=5cm]{doge.jpg}
	\caption{Doge hermoso}
	\label{doge}
\end{figure}

\subsection{creamos variables}

Aquí creo una variable
\def\dogesize {5cm}

\begin{figure}[H]
	\centering
	\includegraphics[width=\dogesize]{doge.jpg}
	\caption{Doge hermoso variable}
\end{figure}

\begin{figure}[H]
	\centering
	\includegraphics[width=\dogesize]{doge.jpg}
	\caption{Doge hermoso variable}
\end{figure}

\begin{figure}[H]
	\centering
	\includegraphics[width=\dogesize]{doge.jpg}
	\caption{Doge hermoso variable}
\end{figure}

\begin{figure}[H]
	\centering
	\includegraphics[width=\dogesize]{doge.jpg}
	\caption{Doge hermoso variable}
\end{figure}

\subsection{usamos las funciones}

\def\dogechico {1cm}
\insertadoge{\dogechico}{1}
\insertadoge{\dogechico}{2}
\insertadoge{\dogechico}{3}
\insertadoge{\dogechico}{4}
\insertadoge{\dogesize}{5}

\subsection{la cosa se complica}

\insertaLegionDeDoges{1cm}

\section{ecuaciones}

$a=b$ \\
$\int_{a}^{b} f(x) = 5$ \\
$\frac{a}{b}$ \\

\section{tablas}

% Table generated by Excel2LaTeX from sheet 'Hoja1'
\begin{table}[htbp]
	\centering
	\caption{Mi tabla sin saber latex}
	\begin{tabular}{|c|c|c|c|c|}
		\hline
		\textbf{a} & \textbf{b} & \multicolumn{2}{c|}{\textbf{c}} & \textbf{e} \bigstrut\\
		\hline
		1     & 2     & 3     & 4     & 5 \bigstrut\\
		\hline
		pedro & juan  & diego & marcel & claudio \bigstrut\\
		\hline
		hola  & casa  & doge  & aburrido & papu \bigstrut\\
		\hline
	\end{tabular}%
	\label{tab:addlabel}%
\end{table}%

Este es un ejemplo de párrafo, no tiene mucho la verdad. Soy super malo escribiendo cualquier cosa.




Como se pueden dar cuenta % esto es un comentario
los saltos de linea no importan en el código de latex. Si uno quiere forzar un nuevo párrafo debe usar doble backslash. \\

Esto sí es un párrafo nuevo.

\section{Esto es una sección}

\section{Y esto, una subsección}

Acá debería escribir algo interesante.

\section*{Y esto una sección sin numerar}

\subsection{Jelou}

\subsubsection{Esto puede ser eterno...}
Hola

\newpage
\section{Realizando operaciones matemáticas con Latex}

% Creamos una funcion que multiplique dos valores
\newcommand{\mult}[3]{
	\MULTIPLY{#1}{#2}{#3}
}

% Calculamos algunas multiplicaciones
\mult{\numberPI}{6}{\seisPI}
\mult{2}{4}{\doge}

Y así, doge=\doge\ y $6\pi$=\seisPI. \\

% Crea la funcion t -> exp(-t)
\SCALEVARIABLEfunction
{-1}{\EXPfunction}
{\NEGEXPfunction}


% Crea la funcion exp(-t)cos(t)
\PRODUCTfunction
{\NEGEXPfunction}
{\COSfunction}
{\NEGEXPCOSfunction}

% Crea la funcion 3t^2-2exp(-t)cos(t)
\LINEARCOMBINATIONfunction
{3}{\SQUAREfunction}
{-2}{\NEGEXPCOSfunction}
{\myfunction}


% Todas las funciones calculan f(t) y f'(t), por lo tanto se requieren dos parámetros para guardar la solución
\myfunction{5}{\sol}{\Dsol}

If
\[
f(t)=3t^2-2\mathrm{e}^{-t}\cos t
\]
then
\[
\begin{gathered}
f(5)=\sol\\
f’(5)=\Dsol
\end{gathered}
\]

\end{document}